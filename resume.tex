%!TEX TS-program = xelatex
%!TEX encoding = UTF-8 Unicode
% Awesome CV LaTeX Template
%
% This template has been downloaded from:
% https://github.com/posquit0/Awesome-CV
%
% Author:
% Claud D. Park <posquit0.bj@gmail.com>
% http://www.posquit0.com
%
% Template license:
% CC BY-SA 4.0 (https://creativecommons.org/licenses/by-sa/4.0/)
%


%%%%%%%%%%%%%%%%%%%%%%%%%%%%%%%%%%%%%%
%     Configuration
%%%%%%%%%%%%%%%%%%%%%%%%%%%%%%%%%%%%%%
%%% Themes: Awesome-CV
\documentclass[]{awesome-cv}
\usepackage{textcomp}
%%% Override a directory location for fonts(default: 'fonts/')
\fontdir[fonts/]

%%% Configure a directory location for sections
\newcommand*{\sectiondir}{resume/}

%%% Override color
% Awesome Colors: awesome-emerald, awesome-skyblue, awesome-red, awesome-pink, awesome-orange
%                 awesome-nephritis, awesome-concrete, awesome-darknight
%% Color for highlight
% Define your custom color if you don't like awesome colors
\colorlet{awesome}{awesome-red}
%\definecolor{awesome}{HTML}{CA63A8}
%% Colors for text
%\definecolor{darktext}{HTML}{414141}
%\definecolor{text}{HTML}{414141}
%\definecolor{graytext}{HTML}{414141}
%\definecolor{lighttext}{HTML}{414141}

%%% Override a separator for social informations in header(default: ' | ')
%\headersocialsep[\quad\textbar\quad]
    \begin{document}
    
%%%%%%%%%%%%%%%%%%%%%%%%%%%%%%%%%%%%%%
%     Profile
%%%%%%%%%%%%%%%%%%%%%%%%%%%%%%%%%%%%%%
\begin{center}
	\headerfirstnamestyle{Anthony} \headerlastnamestyle{McGaw} \\
	\vspace{2mm}
	{\faEnvelope\ ajm279@pitt.edu} | {\faMobile\ 412-228-7901} | {\faMapMarker\ Pittsburgh, PA}% | %{\faLink\ website@website.com}
\end{center}
%%%%%%%%%%%%%%%%%%%%%%%%%%%%%%%%%%%%%%
%     Education
%%%%%%%%%%%%%%%%%%%%%%%%%%%%%%%%%%%%%%

\cvsection{Education}
\begin{cventries}
	\cventry
	{Bachelor of Science in Computer Science}
	{University of Pittsburgh}
	{Pittsburgh, PA}
	{August 2017 – Expected April 2021}
	{GPA: 3.44 \newline
	Dean's List: Fall 2018, Spring 2019, Fall 2019}
	
	
\end{cventries}

\vspace{-2mm}
%%%%%%%%%%%%%%%%%%%%%%%%%%%%%%%%%%%%%%
%     Experience
%%%%%%%%%%%%%%%%%%%%%%%%%%%%%%%%%%%%%%
\cvsection{Experience}
\begin{cventries}
	\cventry
	{Stock Room/Sales Floor/Front End Associate}
	{Homegoods}
	{McCandless, Pa}
	{November 2015– August 2019}
	{\begin{cvitems}
		\item {Helped customers with any issues and always provided highest level of customer service }
		\item {Worked with a team to ensure truck deliveries were unloaded in a timely manner}
		\item {Helped managers reach monthly numbers for TJX credit sales }
		\item {Contributed to maintaining the stores cleanliness as well as safety}
		\end{cvitems}}
\end{cventries}
\vspace{-4mm}
\cvsection{Projects}
\begin{cventries}
	\cventry
	{Accessible at \href{http://www.pitt.edu/~ajm279/} {http://www.pitt.edu/~ajm279/}}
	{Personal Website}
	{Pittsburgh, PA}
	{2020}
	{\begin{cvitems}
		\item {Used Bootstrap open-source framework as a base for the website}
		\item {Designed core layout of website inside an HTML template and styled each section with CSS}
		\item {Optimized layout to work on mobile platforms/smaller windows}
		\end{cvitems}}
	\cventry
	{Accessible at \href{http://pitt-study.appspot.com} {http://pitt-study.appspot.com}}
	{Pitt-Study Web Application}
	{Pittsburgh, PA}
	{2020-work in progress}
	{\begin{cvitems}
		\item {Working with group members to develop a "study session" web application}
		\item {Implemented Python with Flask for back-end event and user creation}
		\item {Application uses Cloud Datastore NoSQL database to store user and event data}
		\item {Utilizes XmlHTTP requests to push changes to user }
		\end{cvitems}}
	\cventry
	{small extra credit assignment to print out binary search tree accurately to command line}
	{BST accurate print function }
	{Pittsburgh, PA}
	{2018}
	{\begin{cvitems}
		\item {Program takes in text files of numbers or letters in any order}
		\item {Program sorts these inputs into an order which will produce a balanced BST }
		\item {Outputs BST to the command line in visually pleasing manner}
		\end{cvitems}}
	%\cventry
	%{java code emulating function of RISCV architecture}
	%{java-assembler}
	%{Pittsburgh, PA}
	%{2018}
	%{\begin{cvitems}
	%	\item {MIPS style instruction assembler written in java}
	%	\item {Utilizes bit shifting and logic functions to isolate bit sections}
	%	\item {Splits pseudo MAL instructions into the appropriate TAL instruction set}
	%	\end{cvitems}}
	\cventry
	{"CHEMICAL LOOPING COMBUSTION: A PIONEER PROCESS TOWARDS CLEAN ENERGY"}
	{First-Year Engineering Conference - Swanson School of Engineering}
	{Pittsburgh, PA}
	{2018}
	{\begin{cvitems}
		\item {Co-wrote semester long research paper on the technology of chemical looping}
		\item {Submitted drafts at different deadlines throughout the process}
		\item {Created detailed presentation and poster to supplement our paper}
		\item {Met with peer groups and section chair to exchange feedback and practice presenting }
		\item {Presented paper in front of fellow classmates, families, and visiting employers }
		\end{cvitems}}
	
	\vspace{-3mm}
\end{cventries}
\cvsection{Skills}
\begin{cventries}
	\cventry
	{}
	{\def\arraystretch{1.15}{\begin{tabular}{ l l }
		Languages:  & {\skill{ Java, C, C++, Python, JavaScript, HTML, CSS}} \\
		Technology:  & {\skill{Node.js, Vue.js, Microsoft Office, Git/Github, UNIX, Google App Engine, Photoshop}} \\
		Coursework:  & {\skill{ Web Applications, OS, Algorithims, Data Structures, Discrete Mathematics,}} \\
		 		    & {\skill{    Computer Organization, Calculus I-III, Written Professional Communication}} \\
		\end{tabular}}}
	{}
	{}
	{}
\end{cventries}

\vspace{-11mm}
\cvsection{Links}
\begin{cventries}
	\cventry
	{}
	{\def\arraystretch{1.20}{\begin{tabular}{ l l }
		Github:		\href{https://github.com/amcgaw07}{\bf 	https://github.com/amcgaw07} \\
		LinkedIn:	\href{https://www.linkedin.com/in/anthony-mcgaw/}{\bf https://www.linkedin.com/in/anthony-mcgaw} \\
		Website:		\href{http://www.pitt.edu/~ajm279/}{\bf 	http://www.pitt.edu/\~{}ajm279/} \\
		\end{tabular}}}
	{}
	{}
	{}
\end{cventries}
\ 
\end{document}